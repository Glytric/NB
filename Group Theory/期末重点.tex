\documentclass{ctexart}
\usepackage{amsmath}
\usepackage{hyperref}
\begin{document}

\title{群论期末重点复习}
\author{严思伟}
\maketitle

\tableofcontents
\newpage

\setcounter{page}{1}

\section{前言}

由于本笔记于1月3日中午12时16分创建,而考试时间为1月7日早8时30分,故实际内容
不会太多,仅会记录部分本人认为较为重要或作业中常出现、考试中出现可能性较大的
内容,尽量捡重点、定理不给出证明只管用。

本课程由五个主要部分组成:绪论和数学基础、群的基本概念、群的线性表示理论、置
换群、三维转动群。其中绪论和数学基础不会作为独立的考试部分出现,故本复习笔记
中将不会包含此部分内容。下面将对每个部分章节的重点内容做一个回顾。

本笔记仅作为复习参考,不能保证我认为的重点内容能够完全匹配考试中的重点内容。
本笔记旨在帮助减小复习强度,给出针对考试的复习内容,如果希望全面整体深入地学
习群论这门课程,\textbf{还是不要看这篇了}。

\section{群的基本概念}

本章主要给出了群的基本概念和乘法表、群的子集、同态同构、点群空间群简介这些内
容。

\subsection{群的定义}

$G$是一些元素的集合,$\forall R$、$S$、$T$为$G$中元素,定义二元运算$RS$,
满足以下四个条件的$G$称为群:

\begin{enumerate}
    \item 封闭性:$RS\in G$
    \item 结合律:$R(ST)=(RS)T$
    \item 恒元:$\exists E\in G, ER=R$
    \item 逆元:$\exists R^{-1}\in G, R^{-1}R=E$
\end{enumerate}

\subsection{有限群部分概念}

\begin{enumerate}
    \item 有限群的阶: 群的元素数目.
    \item 群元素的阶: 对群元素$R$, 使$R^{n}=E$成立的最小正整数n.
    \item 周期: 由群元$R$及其所有幂次构成的集合$\left\{R,R^{2},\ldots,R^{n}
=E \right\}$.
    \item 生成元: 能够通过乘积生成整个群$G$中所有元素的一组最少群元.
    \item 有限群的秩: 生成元的数目.
\end{enumerate}

\subsection{重排定理和乘法表}

\begin{enumerate}
    \item 重排定理:  群中任意元素和群中所有元素做乘积,得到的集合和原群相同.
    \item 乘法表: 有限群的二元运算规则,群的全部性质都体现在群的乘法表中。对
    于有限群,群元素数目有限,我们可能把元素的乘积全部排列出来,构成一个表,
    称为群的乘法表,简称群表。
\end{enumerate}

重排定理在乘法表中的应用:
\begin{enumerate}
    \item 乘法表的每一行 (列)都是所有群元的一个重新排列.
    \item 任一群元素在乘法表中的每一行 (列)中只出现一次.
\end{enumerate}

借助重排定理,我们只需要知道群$G$中少数几个乘积结果就能够直接给出乘法表.

\subsection{子群、陪集和类}

\begin{enumerate}
    \item 子群: $H$是群$G$的子集,且定义有和$G$相同的运算规则,若满足群的四条
    定义,就称$H$是$G$的子群,记为$H\subset G$
    \item 陪集: 设$H$是$G$的子群,$\forall R \notin H$且$R\in G$,则$RH$和
    $HR$分别称为$H$的左陪集和右陪集.
    \item 陪集性质: 陪集$RH$和$H$没有公共元素,且自身没有重复元素.
    \item 陪集定理: $H$的左右陪集$RH$和$HR$,要么拥有完全相同的元素,要么拥有
    完全不同的元素.
    \item 拉格朗日定理: 群$G$的阶$g$一定是子群$H$的阶$h$的整数倍,$g=dh$,$d
    $为正整数,称为子群$H$的指数.
\end{enumerate}

\section{群的线性表示理论}

本章主要围绕有限群的线性表示展开讨论.由于作业和考试内容不涉及诱导分导表示、
物理应用、投影算符、不可约基,故在这里不做复习。复习内容主要有:线性表示基本
概念、正则表示、不等价不可约表示、正交和完备性定理、特征标表。

由于考试题目中不会出现任何证明题,在这里所做的对基本概念的所有回顾均是为了更
好地理解计算过程和加快计算速度,而具体对考试的提升需要结合对作业的回顾,我会
在某一部分需要结合哪些题目复习时具体指出。

\subsection{线性表示基本概念}

\begin{enumerate}
    \item 线性表示: 行列式不为零的$m\times m$阶矩阵集合构成的群$D(G)$和群
    $G$同构,$D(G)$称为群$G$的一个$m$维线性表示,简称表示.每一个群元$R$都对
    应表示矩阵$D(R)$.恒元表示矩阵为单位矩阵,逆元表示矩阵为逆矩阵.
    \item 特征标: 表示矩阵$D(R)$的迹,$\chi(R)=\mathrm{Tr}D(R)$称为$R$在
    表示$D(G)$中的特征标.同类元素特征标相同.
\end{enumerate}

普遍来说,要给出群$G$的线性表示,要满足两个条件:
\begin{enumerate}
    \item 给出基矢$\psi_{\mu}$.
    \item 给出群元在基矢下对应的对称变换算符$P_{G}$.
\end{enumerate}
\noindent 则给出对称变换群元$P_{R}$在基$\psi$中的矩阵形式:
\begin{equation}
    P_{R}\psi_{\mu}=\sum_{\nu=1}^{m}\psi_{\nu}D_{\nu\mu}(R)
\end{equation}
当$P_{R}$和$R$存在一一对应关系时,$D(R)$即是群$G$的线性表示.(其实一多对应也
算,但是这里为了方便理解,毕竟都是复习了,主打一个直接简洁)\\

\noindent 如果取群元$R$作为基矢,也取群元$S$作为算符,那么会给出以下形式的表示:
\begin{equation}
    SR=\sum_{P\in G}PD_{PR}
\end{equation}
此时的$D$即为群$G$的正则表示,由定义可以知道:
\[
D_{PR}=\begin{cases}
1,\quad P=SR & \\
0,\quad P\neq SR &
\end{cases}
\]

右正则表示在此不予列出.

\subsection{等价表示和可约表示}

\begin{enumerate}
    \item 等价表示: 可以由相似变换联系起来的表示.两表示等价的充要条件是对
    所有元素特征标相等.
    \item 对有限群,只需要研究幺正表示和幺正的相似变换.
    \item 一个表示的表示矩阵如果能化成上三角阶梯矩阵,那么就是可约表示,否则
    为不可约表示.
    \item 有限群的表示要么是完全可约表示,要么是不可约表示.
\end{enumerate}

\subsection{不等价不可约表示}

\noindent 正交定理: 不等价不可约幺正表示$D^{i}$和$D^{j}$的矩阵元素,作为群
空间的矢量满足正交关系:
\[
\sum_{R\in G}D^{i}_{\mu\rho}(R)^{*}D^{j}_{\nu\lambda}(R)=\frac{g}{m_{j}}\delta_{ij}\delta_{\mu\nu}\delta_{\rho\lambda}
\]
其中$m_{j}$为$D^{j}$的维数,$g$为群$G$的阶.\\
取$\rho=\mu,\lambda=\nu$,对$\rho\lambda$求和,则有特征标第一正交定理:
\[
\sum_{R\in G}\sum_{\rho\lambda}D^{i}_{\rho\rho}(R)^{*}D^{j}_{\lambda\lambda}(R)=\frac{g}{m_{j}}\delta_{ij}\sum_{\rho\lambda}\delta_{\rho\lambda}\delta_{\rho\lambda}
\]
即
\[
\sum_{R\in G}\chi^{i}(R)^{*}\chi^{j}(R)=g\delta_{ij}
\]

\noindent 完备性定理:有限群不等价不可约表示维数的平方和等于群的阶数:
\[
g=\sum_{j}m_{j}^{2}
\]

\noindent 由此给出有限群不等价不可约幺正表示的矩阵元素$D^{i}_{µν}(G)$,作
为群空间的矢量,构成群空间的正交完备基,任何群函数$f(G)$均可按它们展开:
\[
f(R)=\sum_{j\mu\nu}C^{j}_{\mu\nu}D^{j}_{\mu\nu}(R),\quad C^{j}_{\mu\nu}=\frac{m_{j}}{g}\sum_{R\in G}D^{j}_{\mu\nu}(R)^{*}f(R)
\]
类似傅里叶展开的关系式.\\

\noindent 将矩阵元替换为特征标,群空间替换为类空间,依然成立,说明有限群不等价
不可约表示的特征标$\chi^{j}(G)$构成类空间的正交完备基,任何类函数可以按照下
面的关系对特征标做展开:
\[
f(R)=f(SRS^{-1})=\sum_{j}C_{j}\chi^{j}(R),\quad C_{j}=\frac{1}{g}\sum_{R\in G}\chi^{j}(R)^{*}f(R)
\]
很容易得到有限群不等价不可约表示的个数等于群的类数.

这说明了特征标在群表示理论中的重要地位,也是为什么群表示的特征标表如此重要且
被列为考试重点.

\subsection{可约表示向不可约表示的约化}

对于可约表示,总可以通过相似变换$X$将其变为已约表示,即不可约表示的直和:
\[
X^{-1}D(R)X=\oplus a_{j}D^{j}(R), \quad \chi(R)=\sum_{j}a_{j}\chi^{j}(R)
\]
$a_{j}$为不可约表示$D^{j}(R)$在$D(R)$中的重数.将上面的群函数对特征标做展开中的群函数替换为特征标,可以得到$a_{j}$的表达式:
\[
a_{j}=\frac{1}{g}\sum_{R\in G} \chi^{j}(R)^{*}\chi(R)
\]
表示为不可约表示的判据是:$\sum_{R\in G}\left|\chi(R)\right|^{2}=g$,若大于,则此表示可约.\\

如何将可约表示化为不可约表示:
\begin{enumerate}
    \item 选取一组生成元,写出其在此表示下的表示矩阵.
    \item 计算$\sum_{R\in G}\left|\chi(R)\right|^{2}$,判断表示是否可约.
    \item 若可约,给出此群对应的不可约表示个数及维数.判据有二:不等价不可约表示个数的和为类数;不等价不可约表示维数的平方和为群的阶.
    \item 计算此表示约化后各不等价不可约表示的重数$a_{j}=\frac{1}{g}\sum_{R\in G} \chi^{j}(R)^{*}\chi(R)$,并给出在选择约化的次序下给出之前选取生成元的表示矩阵.
    \item 将各生成元的两个表示矩阵代入$X^{-1}D(R)X=\oplus a_{j}D^{j}(R)$,即可以给出变换矩阵$X$.
    \item 给出基矢变换$\phi_{\mu}=\sum_{\nu}\psi_{\nu}X_{\nu\mu}$
\end{enumerate}

结合第3章作业第17题边做边看食用更佳.

\subsection{新表示构成}
这里仅简单提及,因为不是考试内容但是和特征标表的填写有关.

\begin{enumerate}
    \item 商群的不可约表示也是原群的不可约表示.
    \item 直乘群的不可约表示可以表示为两子群不可约表示的直乘.
\end{enumerate}

\subsection{不可约表示的特征标表}

要顺利写出一个群的特征标表,以下几点性质是必须牢记的:
\begin{enumerate}
    \item 不等价不可约表示的数目为类的个数.
    \item 不等价不可约表示维数的平方和为群的阶.
    \item 恒元的特征标为不可约表示的维数.
    \item 恒等表示的特征标均为1.
    \item 除了恒等表示,每一行特征标乘类中元素数目求和等于0.
    \item 每一列特征标平方和乘类中元素数目等于群的阶.
    \item 
\end{enumerate}






\[
\begin{array}{|c|c|c|c|c|c|}
\hline
a_{11} & a_{12} & a_{13} & a_{14} & a_{15} & a_{16} \\
\hline
a_{21} & a_{22} & a_{23} & a_{24} & a_{25} & a_{26} \\
\hline
a_{31} & a_{32} & a_{33} & a_{34} & a_{35} & a_{36} \\
\hline
a_{41} & a_{42} & a_{43} & a_{44} & a_{45} & a_{46} \\
\hline
a_{51} & a_{52} & a_{53} & a_{54} & a_{55} & a_{56} \\
\hline
a_{61} & a_{62} & a_{63} & a_{64} & a_{65} & a_{66} \\
\hline
\end{array}
\]












\end{document}