\documentclass{ctexart}

\begin{document}

\title{群论期末重点复习}
\author{严思伟}
\maketitle

\tableofcontents
\newpage

\setcounter{page}{1}

\section{前言}

由于本笔记于1月3日中午12时16分创建,而考试时间为1月7日早8时30分,故实际内容
不会太多,仅会记录部分本人认为较为重要或作业中常出现、考试中出现可能性较大的
内容。

本课程由五个主要部分组成:绪论和数学基础、群的基本概念、群的线性表示理论、置
换群、三维转动群。其中绪论和数学基础不会作为独立的考试部分出现,故本复习笔记
中将不会包含此部分内容。下面将对每个部分章节的重点内容做一个回顾。

本笔记仅作为复习参考,不能保证我认为的重点内容能够完全匹配考试中的重点内容。
本笔记旨在帮助减小复习强度,给出针对考试的复习内容,如果希望全面整体深入地学
习群论这门课程,\textbf{还是不要看这篇了}。

\section{群的基本概念}

本章主要给出了群的基本概念和乘法表、群的子集、同态同构、点群空间群简介这些内
容。

\subsection{群的定义}

$G$是一些元素的集合,$\forall R$、$S$、$T$为$G$中元素,定义二元运算$RS$,
满足以下四个条件的$G$称为群:

\begin{enumerate}
    \item 封闭性:$RS\in G$
    \item 结合律:$R(ST)=(RS)T$
    \item 恒元:$\exists E\in G, ER=R$
    \item 逆元:$\exists R^{-1}\in G, R^{-1}R=E$
\end{enumerate}

\subsection{有限群部分概念}

\begin{enumerate}
    \item 有限群的阶: 群的元素数目.
    \item 群元素的阶: 对群元素$R$, 使$R^{n}=E$成立的最小正整数n.
    \item 周期: 由群元$R$及其所有幂次构成的集合$\left\{R,R^{2},\ldots,R^{n}
=E \right\}$.
    \item 生成元: 能够通过乘积生成整个群$G$中所有元素的一组最少群元.
    \item 有限群的秩: 生成元的数目.
\end{enumerate}

\subsection{重排定理和乘法表}

\begin{enumerate}
    \item 重排定理:  群中任意元素和群中所有元素做乘积,得到的集合和原群相同.
    \item 乘法表: 有限群的二元运算规则,群的全部性质都体现在群的乘法表中。对
    于有限群,群元素数目有限,我们可能把元素的乘积全部排列出来,构成一个表,
    称为群的乘法表,简称群表。
\end{enumerate}

重排定理在乘法表中的应用:
\begin{enumerate}
    \item 乘法表的每一行 (列)都是所有群元的一个重新排列.
    \item 任一群元素在乘法表中的每一行 (列)中只出现一次.
\end{enumerate}

借助重排定理,我们只需要知道群$G$中少数几个乘积结果就能够直接给出乘法表.

\subsection{子群、陪集和类}

\begin{enumerate}
    \item 子群: $H$是群$G$的子集,且定义有和$G$相同的运算规则,若满足群的四条
    定义,就称$H$是$G$的子群,记为$H\subset G$
    \item 陪集: 设$H$是$G$的子群,$\forall R \notin H$且$R\in G$,则$RH$和
    $HR$分别称为$H$的左陪集和右陪集.
    \item 陪集性质: 陪集$RH$和$H$没有公共元素,且自身没有重复元素.
    \item 陪集定理: $H$的左右陪集$RH$和$HR$,要么拥有完全相同的元素,要么拥有
    完全不同的元素.
    \item 拉格朗日定理: 群$G$的阶$g$一定是子群$H$的阶$h$的整数倍,$g=dh$,$d
    $为正整数,称为子群$H$的指数.
\end{enumerate}




\end{document}